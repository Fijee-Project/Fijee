In the medical field, Magnetic Resonance Imaging scanners (MRI) can acquire data representing a portion of the human body. This MRI data take the form of a DICOM (Digital Imaging and Communications in Medicine) series of images, representing slices of the body. For our use those images are grayscale. The intensity of a pixel depends on the density of water in the tissue. These series of successive images form a three-dimensional image.
